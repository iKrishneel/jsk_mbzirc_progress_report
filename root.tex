
%%%%%%%%%%%%%%%%%%%%%%%%%%%%%%%%%%%%%%%%%%%%%%%%%%%%%%%%%%%%%%%%%%%%%%%%%%%%%%%%

%        1         2         3         4         5         6         7         8

\documentclass[letterpaper, 10 pt, conference]{ieeeconf}  % Comment this line out if you need a4paper

%\documentclass[a4paper, 10pt, conference]{ieeeconf}      % Use this line for a4 paper

\IEEEoverridecommandlockouts                              % This command is only needed if 
                                                          % you want to use the \thanks command

\overrideIEEEmargins                                      % Needed to meet printer requirements.


\usepackage{graphics} % for pdf, bitmapped graphics files
\usepackage{epsfig} % for postscript graphics files
\usepackage{mathptmx} % assumes new font selection scheme installed
\usepackage{times} % assumes new font selection scheme installed
\usepackage{amsmath} % assumes amsmath package installed
\usepackage{amssymb}  % assumes amsmath package installed

\usepackage[font=small,labelfont=bf]{caption}

% Other package
\usepackage{tikz}
\usepackage{graphicx}
\usepackage{caption} 
\usepackage{subcaption}
\usepackage{multirow}
\usepackage{array}
\usepackage{booktabs}
\usepackage{hyperref}

\usepackage{pdfpages}
\usepackage{caption}
%\usepackage{geometry}
\usepackage{import}
\usepackage{standalone}

\title{\LARGE \bf
  Team-JSK: MBZIRC Progress Report
}
\author{Team JSK$^\dagger$% <-this % stops a space
  \\ JSK Lab, Graduate School of Information Science and Technology, The University of Tokyo \\
  7-3-1 Hongo, Bunkyo-ku, Tokyo 113-8656, Japan.  \\
{\tt\small mbzirc-team@jsk.imi.i.u-tokyo.ac.jp}
\thanks{$^{*}$ %JSK Lab, Graduate School of Information Science and Technology, The University of Tokyo,  7-3-1 Hongo, Bunkyo-ku, Tokyo 113-8656, Japan.  
{$^\dagger$\tt\small \url{http://www.jsk.t.u-tokyo.ac.jp}}
}}
\begin{document}

\maketitle
\thispagestyle{empty}
\pagestyle{empty}


\section{Introduction}
This document provides a report of Team JSK’s progress in preparing for the Mohamed Bin Zayed International Robotics Challenge (MBZIRC). The team consists of members from the JSK Laboratory at the University of Tokyo. The JSK Lab, founded in early 1980’s, has a long history of robotics research with focus on areas including humanoids, drones, robotics manipulation, and perception. The lab has experience in participating in robotics challenges including the DARPA Robotic Challenge and the Amazon Picking Challenge.

\section{Project Personnel}
Team JSK is made of 11 members. The 11 members are Prof. Inaba, Prof. Okada, Dr. Kakiuchi, Dr. W. Chan, Bakui Chou, Xiangyu Chen, Krishneel Chaudhary, Kohei Kimura, Yuki Furuta and Hiroto Mizohana. The team is divided into 3 groups generally to handle various tasks such as software and hardware developments and other project related activities. All team members are from JSK Lab, in The University of Tokyo.

%That is a professor, a associate-professor, a lecturer, a researcher, 4 Phd students and 2 Master students. 
%Teachers mainly focus on manage the whole team schedule, design hardware and software architecture, provide advices and give ideas for all the tasks. The rest researcher and students are divided into three groups, 
%one for computer vision development for all the tasks, one for task 2 and one for task 1 and 3.



\section{CHALLENGE 1: LANDING UAV ON A MOVING VEHICLE}


%Copyright (C) 2016 by Krishneel@JSK Lab, The University of Tokyo

\documentclass{standalone}
\usepackage{footnote}
\usepackage{hyperref}
\usepackage{graphicx}

\begin{document}

\subsection{Software}

The softwares are developed on ROS environment which includes motion planning, visual perception and virtual simulation. The simulation environment was developed on Gazebo simulator \footnote{\url{https://github.com/start-jsk/jsk_mbzirc}}. We used this simulator for planning our stretegy and for customization of our hardwares and softwares. The visual perception node involves target (heliport) localization on the moving vehicle, planning our efficient approaching and landing stretegy based on the motion of both the UAV and the vehicle.We use a robust tracking algorithm with efficient tracking drift compensation algorithm to avoid lost of target when the uav is in motion. Moreover our visual tracking algorithm is able to recover the target even if its completely out of view. 

The future work on software for task 1 involves testing the complete software on the actual UAV which is developed by the team. Considering the challenges on the outdoor environment such as abrupt changes in image space, winds speeds etc., we believe that we will have to improve our current landing stretegy in the real world. 

\begin{figure}
  \includegraphics[width=\columnwidth, trim={2cm 5cm 3cm 7cm},clip]{sections/task1/images/testbed}
  \caption{Testbed for outdoor testing in Hachioji, Tokyo, Japan}
\end{figure}


\subsection{General Approach}

\textbf{Vehicle And Heliport Detection}: We use the heliport model to train a linear SVM classifier for detection of landing region. Since heliport model will be consistent it can be used as a priori for learning. Once the heliport is detected, a visual object tracker running at real time onboard is autonomously initalized to start tracking the target region. 

\textbf{}


\subsection{Future Works}

\end{document}

%Copyright (C) 2016 by Krishneel@JSK Lab, The University of Tokyo

\documentclass{standalone}
\begin{document}

\subsection{Hardwares}

We developed the uav with hex rotors as shown in Fig.\ref{fig:task1-uav}. As descrbied in  Fig.\ref{fig:task1-uav-platform}, this aerial robot consists of onboard sensors such as IMU, barometer, laser sensor, GPS for basis hovering flight control, as well as the original flight controller and high level processor. The monocular camera is installed for the further egomotion estimation. The total weight of the uav is 4.3Kg, while the flight time can reach 20min with heavy vision processing on the onboard processor. We have achieved the outdoor flight with autonomous altitude hold mode using our original sensor fusion algorithm(Fig.\ref{fig:task1-outdoor-flight}).

\begin{figure}[h]
  \begin{center}
    \includegraphics[clip,  bb=0 0 710 502,  width=\columnwidth]{sections/task1/images/task1-tarrot680.pdf}
    \caption{Image of task1 UAV(Hawk)}
    \label{fig:task1-uav}
  \end{center}
\end{figure} 

\begin{figure}[h]
  \begin{center}
    \includegraphics[clip,  bb=0 0 720 500,  width=\columnwidth]{sections/task1/images/hawk-platform.pdf}
    \caption{Hardware platform of "Hawk"}
    \label{fig:task1-uav-platform}
  \end{center}
\end{figure} 

\begin{figure}[h]
  \begin{center}
    \includegraphics[clip,  bb=88 84 599 440,  width=\columnwidth]{sections/task1/images/task1-outdoor.pdf}
    \caption{Image of outdoor flight }
    \label{fig:task1-outdoor-flight}
  \end{center}
\end{figure} 



\end{document}

%% insert yout latex module file here. the contents should go to the tasks folder under section

\section{CHALLENGE 2: OPERATING A VALVE STEM}

%Copyright (C) 2016 by Krishneel@JSK Lab, The University of Tokyo

\documentclass{standalone}

\usepackage{hyperref}
\usepackage{footnote}
\usepackage{graphicx}

\begin{document}

\subsection{Hardware}
Our robot consists of an upper body humanoid on a high-power mobile base. The robot is equipped with a stereo camera, a long range laser sensor, a global positioning system (GPS), and a custom made gripper. The gripper consists of a magnet embedded link actuated by a servo motor as shown in Fig.\ref{fig:figure1}. The wrist is also equipped with a six axis force torque sensor. 


\begin{figure}
  \includegraphics[height=2.5in, width=1.5in]{sections/task2/images/figure1}
  \caption{HRP2 robot platform for task2}
  \label{fig:figure1}
\end{figure}

\subsection{Software}

For task 2, the softwares are also implemented on ROS environment with utilization of multithreading for fast computation. Euslisp programming language from JSK lab was used for kinematics simulation and robot control. OpenCV and in-house developed algorithms
 \footnote{\url{https://github.com/jsk-ros-pkg/jsk_recognition}} are used for recognition and perception.


 \subsection{General Approach}
 \textbf{Navigation}: Our approach is to use the long range laser sensor and the GPS for searching and navigating to the panel when the robot is far from the robot and is out of range for the stereo camera. As the panel becomes closer than the minimum range of the laser sensor, the robot will then switch over to use the stereo camera. Our high-powered mobile base can reach up to 4m/s and can drive through various outdoor terrains. 

\textbf{Wrench and valve stem detection}: We experimented and compared infrared camera with stereo camera, and we have decided to use stereo camera for close range perception, since infrared cameras tend to fail in outdoor sunny environments, and cannot sense objects that are too close to the robot. 

We detect the wrench by using Edge detection, Hough transform and K-means. Firstly, we select a region include 6 wrenches on the camera image(Figure 1A). We apply hough line transform to the edge image of selected region and extract lines which have large slope, then apply K-means(K is 6) to extracted lines. The size of wrenches can be estimated from classified lines, but the position of wrenches estimated from lines is not correct(Figure 1B). Therefore we use Hough circle transform to detect the hanger on image(Figure 1C) and get depth from point cloud(Figure 1D). Thus, we can detect the size and position of wrenches.  

We also select the region on the camera image to detect the valve stem(Figure 1A). We estimate the plane from the point cloud in the selected region, extract the points witch exist in front of the plane(Figure 1E). The centoroid of the extracted points is considered as the position of the valve stem(Figure 1F).

\textbf{Picking the wrench}: Our robot picks up the wrench by aligning the gripper with the wrench, moving the gripper toward the wrench, and letting the magnets pull the wrench into the gripper. The gripper is designed so that the wrench is grasps firmly, but with some movement possible for passive compliance.

\textbf{Wrench fitting and turning}: To fit the wrench head onto the valve stem, we use force feedback from the wrist to gauge the tool contact state. The robot first moves its gripper above the detected position of the valve stem. Due to error in detection or calibration, the wrench head could be directly above the valve stem or it can be slightly misaligned (Figure 1A, B). The robot moves its gripper down, until a force in the vertical direction is detected, indicating that the wrench has come in contact with the valve stem. Once it detects contact with the valve stem, the wrench can be in one of the contacts states as shown in Figure 1 The robot then moves its gripper in the horizontal direction in a widening zigzag pattern. Depending on the forces it detects, the robot then begins to turn the wrench, or adjusts its gripper positon and retries to fit the wrench (Figure 1).

\begin{figure}
  \includegraphics{sections/task2/images/figure2}
  \caption{Custom made magnetic gripper.}
  \label{fig:figure2}
\end{figure}


\begin{figure}
  \includegraphics[width=\columnwidth]{sections/task2/images/figure3}
  \caption{Using force fitting for wrench fitting}
  \label{fig:figure3}
\end{figure}


\subsection{Results Achieved to Date}
We have completed the prototypes of our mobile base and customized gripper. The entire robot has been assembled and all sensors are functional. The robot can be operated through teleoperation, and we have been able to successfully complete challenge 2 using full teleoperation indoors and outdoors. 
Recognition of the wrenches and the valve stem has also been implemented. Once we select the region to detect wrenches and valve stem, they will be detected autonomically.
We experimented with wrench fitting and turning with different initial wrench alignments. Among twenty trials, we were able to achieve a 95$\%$ success rate with only one failed trial. 
We have also tested our system for performing the entire challenge 2 with partial autonomy in outdoor experiments. In our experiments, we used teleoperation to drive the robot’s mobile base, allowed the robot to detect the wrenches and valve stem with human supervision, and grasp, fit, and turn the wrench with full autonomy. In our fastest run, we were able to complete challenge 2 in less than ten minutes. This time can be easily shortened as many parts of our code had deliberate pauses for debugging and testing purposes.

\subsection{Future Plans}
Our future plans include speeding up our task completion time, enabling autonomous navigation of the mobile base, autonomous search of the panel, full autonomous wrench and valve stem detection, and failure detection when grasping, fitting, and turning the wrench. We will also consider and compare alternative wrench detection, and wrench fitting methods. Currently, the valve stem we have been operating has very little resistance. While we have successfully turned a valve stem with 5Nm resistance, our gripper prototype broke after turning a quarter turn. We have already strengthened our gripper design, and as future work, we will be testing with valve stems having higher torque resistance.  Finally, we are also considering the potential use of another robot platform that is more lightweight and allows us to more easily transport it from Japan to the competition venue. 

\end{document}

\section{CHALLENGE 3: SEARCH, PICK AND PLACE}



%Copyright (C) 2016 by Krishneel@JSK Lab, The University of Tokyo

\documentclass{standalone}
\begin{document}

\section{task3}
\subsection{Platforms}
For task 3, we applied two kinds of UAVs to challenge the task. The general UAV called "hawk" as shown in Fig.\ref{fig:task3-hawk}, which is similar to the one used in task 1, and the transformable aerial robot with multilink which is called "Hydrus"(Fig.\ref{fig:task3-hydrus}). As described in Fig.\ref{fig:task3-hydrus-platform},tThe hardware platform of "Hydrus" envolves the controller for joints which enables the stable aerial transformation.

\begin{figure}[h]
  \begin{center}
    \includegraphics[clip,  bb=115 4 666 535,  width=\columnwidth]{sections/task3/images/task3-tarrot810.pdf}
    \caption{Image of task3 Hawk}
    \label{fig:task3-hawk}
  \end{center}
\end{figure} 

\begin{figure}[h]
  \begin{center}
    \includegraphics[clip,  bb=0 105 720 535,  width=\columnwidth]{sections/task3/images/task3-hydrus.pdf}
    \caption{Image of Hydrus}
    \label{fig:task3-hydrus}
  \end{center}
\end{figure} 

\begin{figure}[h]
  \begin{center}
    \includegraphics[clip,  bb=0 0 720 540,  width=\columnwidth]{sections/task3/images/hydrus-platform.pdf}
    \caption{Hardware platform of task3 Hawk}
    \label{fig:task3-hydrus-platform}
  \end{center}
\end{figure} 



Although the flight control algorithms between "Hawk" and "Hydrus" are fundamentally different, we use the smae flight controller board which is build by ourselves. We additionally designed another PCB board for controlling the eletromagnet module which can generate the suction force up to 20[N]. We equipped 5 eletromagnet in the UAV and build the attachment with tactile sensors as shwon in Fig.\ref{fig:task3-hawk}(c). The electro-magnet moudle control board is connected to the flight  controller board unit through CAN bus.

For the transformable UAV, we introduce the prototype which contains four links and three servo joints. The modularization of the whole platform is achieved by distributing the power and control system to each link with excpect of flight controller and sensors. Therefore, it becomes easier to the change the amount of rotors, according to the application of the flight.

\subsection{Aerial Manipulation Strategy}
For each type of UAV, we develop different piccking method. For "hawk" type UAV, we appy the magnetic force to absorb the ferrous object as shown in Fig.\ref{fig:task3-hawk-manipulation}. When the contact between the bottom of landing gear and object occurs, the tactile sensor provides certain signal, leading the actication of the eletromagnet module. We have achieve to the pick and carry the object inder indoor enviroment using motion capture system, which confirm the validty of the eletro-magnet based manipulation strategy. The cylinder type object is created according to the regulation description.

\begin{figure}[h]
  \begin{center}
    \includegraphics[clip,  bb=0 110 720 540,  width=\columnwidth]{sections/task3/images/task3-tarrot810-manipulation.pdf}
    \caption{Aerial manipulation method of Hawk}
    \label{fig:task3-hawk-manipulation}
  \end{center}
\end{figure} 

On the other hand, the object transporation based on the whole-body-manipulation strategy using "Hydrus" is also acheived as shown in Fig.\ref{fig:task3-hydrus-manipulation}. The grasping control is developped  based on the torque feedback from each joint. 

\begin{figure}[h]
  \begin{center}
    \includegraphics[clip,  bb=0 0 720 540,  width=\columnwidth]{sections/task3/images/task3-hydrus-manipulation.pdf}
    \caption{Aerial manipulation method of Hydrus}
    \label{fig:task3-hydrus-manipulation}
  \end{center}
\end{figure} 

\subsection{Software}
Just like other tasks the softwares are build on ROS environment and some functionalities are shared from task 1. Point Cloud and OpenCV libraries are used for visual perception. % target detection and motion planning are different.

\subsubsection{General Approach}
%The software system is based on ROS(Robot Operation System). We write our algorithm to the every single node and communicate with each node. 
Basically for task 3, we divide the task into three states: Search, Pick and Place. The UAVs are always within these three states and the states automatically transferred to the next one if the certain condition is satisfied as illustrated in Fig. \ref{t3}A. In "Search" state, the drone will traverse to the center of the arena and randomly generate a search end-point, the treasure detector will work when the drone is searching, once the object is detected and locked, a pick motion will be generated in the "Pick" state, the UAV will open the Elec-Magnet and moving approach to the treasure. The transfer state signal depends on the trigger of the tactile sensor, once the Elec-Magnet catch the treasure, the UAV enters "Place" state, it will directly fly to the placing zone and find the box to place the treasure. After release the treasure upon the place box, the UAV re-enter the "Search" state and loops until task is completed.

 \begin{figure}%[hb]
    \begin{center}
      \includegraphics[keepaspectratio=true, width=1\linewidth, height=0.30\textheight]{img//task3.png}
    \end{center}
    \caption{Task 3 Demonstration}
    \label{t3}
  \end{figure}


\subsubsection{Treasure Detection}
As the treasures have distinct color features compared to the ground, we firstly used a simple detection method to localize the treasure. 
The inputs are 3D points $p_i$ from the Stereo sensor and the RGB image projection to the ground by the projection matrix computed using the known camera parameters. %We first apply 
HSI color filter is applied to obtain to 3D point candidates of treasure from the point cloud data. Next we apply Euclidean clustering to the filtered point cloud $P_{hsi}$. Euclidean clustering technique can organize points into clusters with respect to the distance feature in 3D space. For $\forall p_i, p_j \in P_{hsi}$, clusters $O_i = \{p_i \in P_i\}$ and $O_j = \{p_j \in P_j\}$ are obtained by:
\begin{equation}\label{eq3-1}
min||p_i - p_j|| \geq d_{threhold} 
\end{equation}

When we get all the clusters, we apply a simple tracker to every cluster center and as we continue to detect the same cluster over time space, the weight of the tracker is increased to boost the confidence of tracking. For clusters that are not always detected the confidence are slowly decreased and removed from the treasure candidates vector. The UAV will lock the cluster candidate when the weight is large enough and switch into the "Pick" mode to approach the treasure.

\subsubsection{Simulation}
We first perform full automatic simulation in gazebo environment as shown in Fig.\ref{t3}B. To fully simulate the real scene, we add noise and outliers to the detection. In simulation, the UAV takes almost 70$seconds$ to detect, pick and place a single object. In future we will use three UAVs in coordination to complete the task which will not only decrease the time but also can be used to transport larger treasures which a single UAV might not be able to lift.
% we believe we can do that better. 
For real robot, we tested with tele-operation control, both Hawk and transformable UAV can grasp the treasure, pick and place into a specificed box. We are planing to perform more test on the real robot to justify the detection and motion planning algorithm in the simulation as part of the future work.


\subsection{Future Plan}
The future work on hardware platform for task 3 contains the improvement of the structural strength, as well as the enhancement of the modularization of link system by using CAN communication network. We will also continue to validate the performance of the eletromagnet module, and the  collaboration between the eletromagnetic force and whole-body-manipulation will be developped for the "Hydrus."

The future work on software for task 3 involves the outdoor expeirment with acutal robot to test the performance of both eletromagnet module and whole-body-manipulation. Further more, we will focus on the collaborative motion for picking up the large object using two or three UAV simultaneously, as well as the swarm control strategy while searching object.


\end{document}


\section{GRAND CHALLENGE}


\section{Future Plan}

\end{document}