
%Copyright (C) 2016 by Krishneel@JSK Lab, The University of Tokyo

\documentclass{standalone}
\usepackage{footnote}
\usepackage{hyperref}

\begin{document}

\subsection{Software}

The softwares are developed on ROS environment which includes motion planning, visual perception and virtual simulation. The simulation environment was developed on Gazebo simulator \footnote{visit: \url{https://github.com/start-jsk/jsk_mbzirc.git}}. We used this simulator for planning our stretegy and for customization of our hardwares and softwares. The visual perception node involves target (heliport) localization on the moving vehicle, planning our efficient approaching and landing stretegy based on the motion of both the UAV and the vehicle.We use a robust tracking algorithm with efficient tracking drift compensation algorithm to avoid lost of target when the uav is in motion. Moreover our visual tracking algorithm is able to recover the target even if its completely out of view. 

The future work on software for task 1 involves testing the complete software on the actual UAV which is developed by the team. Considering the challenges on the outdoor environment such as abrupt changes in image space, winds speeds etc., we believe that we will have to improve our current landing stretegy in the real world. 



\end{document}